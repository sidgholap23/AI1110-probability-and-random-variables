\let\negmedspace\undefined
\let\negthickspace\undefined
\documentclass[journal,12pt,twocolumn]{IEEEtran}

\title{Assignment-2}
\author{AI1110 Probability And Random Variables\\Gholap Siddhesh Ashok \\ AI22BTECH11007}

\usepackage{tfrupee}
\usepackage{cite}
\usepackage{amsmath,amssymb,amsfonts,amsthm}
\usepackage{algorithmic}
\usepackage{graphicx}
\usepackage{textcomp}
\usepackage{xcolor}
\usepackage{txfonts}
\usepackage{listings}
\usepackage{enumitem}
\usepackage{mathtools}
\usepackage{gensymb}
\usepackage[breaklinks=true]{hyperref}
\usepackage{tkz-euclide} % loads  TikZ and tkz-base
\usepackage{listings}

\DeclareMathOperator*{\Res}{Res}
%\renewcommand{\baselinestretch}{2}
\renewcommand\thesection{\arabic{section}}
\renewcommand\thesubsection{\thesection.\arabic{subsection}}
\renewcommand\thesubsubsection{\thesubsection.\arabic{subsubsection}}

\renewcommand\thesectiondis{\arabic{section}}
\renewcommand\thesubsectiondis{\thesectiondis.\arabic{subsection}}
\renewcommand\thesubsubsectiondis{\thesubsectiondis.\arabic{subsubsection}}

% correct bad hyphenation here
\hyphenation{op-tical net-works semi-conduc-tor}
\def\inputGnumericTable{}                                 %%

\lstset{
%language=C,
frame=single,
breaklines=true,
columns=fullflexible
}

\begin{document}
%


\newtheorem{theorem}{Theorem}[section]
\newtheorem{problem}{Problem}
\newtheorem{proposition}{Proposition}[section]
\newtheorem{lemma}{Lemma}[section]
\newtheorem{corollary}[theorem]{Corollary}
\newtheorem{example}{Example}[section]
\newtheorem{definition}[problem]{Definition}
\newcommand{\BEQA}{\begin{eqnarray}}
\newcommand{\EEQA}{\end{eqnarray}}
\newcommand{\define}{\stackrel{\triangle}{=}}

\bibliographystyle{IEEEtran}
\providecommand{\mbf}{\mathbf}
\providecommand{\pr}[1]{\ensuremath{\Pr\left(#1\right)}}
\providecommand{\qfunc}[1]{\ensuremath{Q\left(#1\right)}}
\providecommand{\sbrak}[1]{\ensuremath{{}\left[#1\right]}}
\providecommand{\lsbrak}[1]{\ensuremath{{}\left[#1\right.}}
\providecommand{\rsbrak}[1]{\ensuremath{{}\left.#1\right]}}
\providecommand{\brak}[1]{\ensuremath{\left(#1\right)}}
\providecommand{\lbrak}[1]{\ensuremath{\left(#1\right.}}
\providecommand{\rbrak}[1]{\ensuremath{\left.#1\right)}}
\providecommand{\cbrak}[1]{\ensuremath{\left\{#1\right\}}}
\providecommand{\lcbrak}[1]{\ensuremath{\left\{#1\right.}}
\providecommand{\rcbrak}[1]{\ensuremath{\left.#1\right\}}}
\theoremstyle{remark}
\newtheorem{rem}{Remark}
\newcommand{\sgn}{\mathop{\mathrm{sgn}}}
\providecommand{\abs}[1]{$\left\vert#1\right\vert$}
\providecommand{\res}[1]{\Res\displaylimits_{#1}}
\providecommand{\norm}[1]{$\left\lVert#1\right\rVert$}
\providecommand{\mtx}[1]{\mathbf{#1}}
\providecommand{\mean}[1]{$E\left[ #1 \right]$}
\providecommand{\fourier}{\overset{\mathcal{F}}{ \rightleftharpoons}}
\providecommand{\system}{\overset{\mathcal{H}}{ \longleftrightarrow}}
\newcommand{\solution}{\noindent \textbf{Solution: }}
\newcommand{\cosec}{\,\text{cosec}\,}
\providecommand{\dec}[2]{\ensuremath{\overset{#1}{\underset{#2}{\gtrless}}}}
\newcommand{\myvec}[1]{\ensuremath{\begin{pmatrix}#1\end{pmatrix}}}
\newcommand{\mydet}[1]{\ensuremath{\begin{vmatrix}#1\end{vmatrix}}}

\maketitle

\noindent \textbf{10.15.1.10} A piggy bank contains hundred $50 p$ coins, fifty \rupee $1$ coins, twenty \rupee $2$ coins and ten \rupee $5$
coins. If it is equally likely that one of the coins will fall out when the bank is turned upside down, what is the probability that the coin
\begin{enumerate}[label=(\roman*)]
    \item will be a $50 p$ coin ?
    \item will not be a \rupee $5$ coin?
\end{enumerate}

\noindent \textbf{solution:}
Given:
\begin{align}
    \notag\text{total number of coins} &= 100+50+20+10\\
    &=180
\end{align}

\begin{enumerate}[label=(\roman*)]

    \item number of $50 p$ coins = 100\\
           \begin{align}
               \notag \pr{50p} &= \frac{100}{180}\\
               &= \frac{5}{9}
           \end{align}

    \item number of \rupee5 coins = 10\\
           \begin{align}
               \notag \pr{\text{\rupee} 5} &= \frac{180-10}{180}\\
               &= \frac{17}{18}
           \end{align}
           
\end{enumerate}
   
   

\end{document}
